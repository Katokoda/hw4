\documentclass{article}
\usepackage[utf8]{inputenc}

\usepackage[T1]{fontenc}
\usepackage[english]{babel} % Importe deux langues et choisit la deuxième pour maintenant.
%\selectlanguage{english} % Permet de changer de langue au milieu du document.
\usepackage{lmodern}

%\usepackage[framed,numbered,autolinebreaks,useliterate]{mcode}

\usepackage{amsmath}
\usepackage{amssymb}
\usepackage{amsthm}
%\usepackage{parskip} 		%noindent
\usepackage{xfrac} 			%\sfrac{1}{2}
\usepackage{verbatim} 		% \begin{comment}
\usepackage{stackengine} 	% To define \circled later
\usepackage{enumitem}
\usepackage{cancel}			% \cancel{texte barré} %\bcancel, \xcancel


\usepackage{geometry}
\geometry{hmargin=2.3cm,vmargin=3cm}  % changer les marges \textbfhorizontales et verticales

\newcommand{\R}{\mathbb{R}}
\newcommand{\Rn}{\mathbb{R}^n}
\newcommand{\Q}{\mathbb{Q}}
\newcommand{\Z}{\mathbb{Z}}
\newcommand{\N}{\mathbb{N}}
\newcommand{\E}{\mathcal{E}}

\newcommand{\fl}{f_\lambda}
\newcommand{\dej}{\dfrac{\partial}{\partial x_j}}

\newcommand{\sump}{\sum_{i=1}^p}
\newcommand{\summ}{\sum_{i=1}^m}
\newcommand{\sumN}{\sum_{i=1}^N}
\newcommand{\x}{x_ {k+1}}
\newcommand{\e}{\eta_k}
\newcommand{\I}{\mathcal{I}}
\renewcommand{\Im}{\operatorname{Im}}
\newcommand{\T}{\text{T}}
\newcommand{\F}{\text{F}}
\newcommand{\Proj}{\text{Proj}}

%%%%%%%%%%%%%%%%%%%%%%%%%%%%%%%%%%%%%%%%%

\title{Homework 4 - Continuous Optimization}
\author{\xcancel{Estelle} Ezra Baup, Samuel Bélisle, Cassandre Renaud }

\begin{document}
\maketitle

holà les gus \\
ce devoir, Sam'enchante pas, vous?\\
(un peu cass-é comme blague je l'avoue, mais gardez ez-poir en moi please)

\section*{Question 1}

NOTE/TODO je suis pas du tout sure que cette notation est très belle ducoup dites si vous voulez changer. Aussi c'est très moche les produit scalaires "horizontaux" mais plus lisible? jsp\\

The feasible set is $S=\{x,y \in \R^n | h(x,y)=0\}=\{x,y \in \R^n | 1-x^\top x=0, 1-y^\top y=0, x^\top y=0 \}$. \\
It is not convex. Indeed, we will give two points $z_1$ and $z_2$ $\in S$, but such that $z=\lambda z_1 + (1-\lambda) z_2 \notin S$ for a given $\lambda$. We will work with these $z_i \in \R^2 \times \R^2$ i.e. $n=2$.\\


\noindent We will take $z_1=(x_1,y_1)=((1,0),(0,1))$. First we check that $z_1 \in S$.
\begin{itemize}
\item $1-x_{1}^\top x_1=1-\left<(1,0),(1,0)\right>=1-1=0$
\item $1-y_{1}^\top y_1=1-\left<(0,1),(0,1)\right>=1-1=0$
\item $x_{1}\top y_1=\left<(1,0),(0,1)\right>=0$
\end{itemize}
And we will take $z_2=(x_2,y_2)=((0,1),(1,0))$. we also check that $z_2 \in S$.
\begin{itemize}
\item $1-x_{2}^\top x_2=1-\left<(0,1),(0,1)\right>=1-1=0$
\item $1-y_{2}^\top y_2=1-\left<(1,0),(1,0)\right>=1-1=0$
\item $x_{2}\top y_2=\left<(0,1),(1,0)\right>=0$
\end{itemize}
Lastly, we will take $\lambda=\frac{1}{2}$. Now we can compute our $z=\lambda z_1 + (1-\lambda) z_2$
$$z=\lambda z_1 + (1-\lambda) z_2=\frac{1}{2}((1,0),(0,1))+\frac{1}{2}((0,1),(1,0))=((\frac{1}{2},\frac{1}{2}),(\frac{1}{2},\frac{1}{2}))=(x,y)$$
But if we compute $x^\top y=\left< (\frac{1}{2},\frac{1}{2}),(\frac{1}{2},\frac{1}{2}) \right>=(\frac{1}{2})^2+(\frac{1}{2})^2=\frac{1}{2} \neq 0$, so the third condition of our function $h$ does not hold on this point, hence our set is not convex.

\section*{Question 2}

\section*{Question 3}

\section*{Question 4}

\section*{Question 5}

\section*{Question 6}

\section*{Question 7}

\section*{Question 8}














\end{document}